%!TEX TS-program = xelatex
%!TEX encoding = UTF-8 Unicode

% biblatex-dw 
% Copyright (c) Dominik Waßenhoven <domwass(at)web.de>, 2013

\documentclass[english]{scrartcl}
%!TEX encoding = UTF-8 Unicode

% biblatex-dw 
% Copyright (c) Dominik Waßenhoven <domwass(at)web.de>, 2016
%
% This file is the preamble for the documentation of 
% biblatex-dw (both  the English and the German version)

%%%%% biblatex-dw Version %%%%% version of biblatex-dw %%%%%
\newcommand{\biblatexdwversion}{1.7}
\newcommand{\biblatexdwdate}{\printdate{2016-12-06}}
\newcommand{\mindestanforderung}{3.3}% minimum biblatex version
\newcommand{\testversion}{3.6}% tested biblatex version
\newcommand{\biberversion}{2.6}% tested biber version
\newcommand{\screenversion}{}
\newcommand{\TOC}{}
\newcommand{\lizenz}{}
	
%%%%% Kodierung %%%%% Encoding %%%%%
\usepackage{fontspec,xltxtra,xunicode}

%%%%% Inhaltsverzeichnis %%%%% Table of Contents %%%%%
\setcounter{tocdepth}{2}

%%%%% Schriftarten %%%%% Fonts %%%%%
\usepackage[osf]{libertine}
\defaultfontfeatures{Mapping=tex-text}
\setmonofont[Scale=MatchLowercase]{Bitstream Vera Sans Mono}
\setkomafont{sectioning}{\sffamily}%     Überschriften
\renewcommand{\headfont}{\normalfont\itshape}% Kolumnentitel

%%%%% Fußnoten %%%%% Footnotes %%%%%
\deffootnote%
  {2em}% Einzug des Fußnotentextes; bei dreistelligen Fußnoten evtl. vergrößern
  {1em}% zusätzlicher Absatzeinzug in der Fußnote
  {%
  \makebox[2em]% Raum für Fußnotenzeichen: ebenso groß wie Einzug des FN-Textes
    [r]% Ausrichtung des Fußnotenzeichens: [r]echts, [l]inks
    {\libertineLF% keine Mediävalziffern als Fußnotenmarke
    \thefootnotemark%
    \hspace{1em}% Abstand zw. FN-Zeichen und FN-Text
    }%
  }
\renewcommand{\footnoterule}{}% keine Zeile zw. Text und Fußnoten

%%%%% Kopf- und Fußzeilen %%%%% page header and footer %%%%%
\usepackage{scrpage2}
\pagestyle{scrheadings}
\clearscrheadfoot

%%%%% Farben %%%%% Colors %%%%%
\usepackage{xcolor}
\definecolor{dkblue}{rgb}{0 0.1 0.5}%			dark blue
\definecolor{dkred}{rgb}{0.85 0 0}%				dark red
\definecolor{pyellow}{rgb}{1 0.97 0.75}%	pale yellow

%%%%% Kompakte Listen %%%%% Compact lists %%%%%
\usepackage{enumitem}
\setlist{noitemsep}
\setitemize{leftmargin=*,nolistsep}
\setenumerate{leftmargin=*,nolistsep}

%%%%% Datumsformat %%%%% Date format %%%%%
\usepackage{isodate}
\numdate[arabic]% 15. 9. 2008
\isotwodigitdayfalse% führende Nullen weglassen

%%%%% Spracheinstellungen %%%%% Language settings %%%%%
\usepackage{babel}
\newcommand{\versionname}{Version}
\iflanguage{english}{\renewcommand{\versionname}{version}}{}
\iflanguage{german}{\monthyearsepgerman{\,}{\,}}{}
\iflanguage{ngerman}{\monthyearsepgerman{\,}{\,}}{}
\iflanguage{english}{\isodate}{}

%%%%% Verschiedene Pakete %%%%% Miscellaneous packages
\usepackage{microtype}%		optischer Randausgleich
\usepackage{dtk-logos}%		Logos wie \BibTeX etc.
\usepackage{xspace}
\usepackage{textcomp}%		Text-Companion-Symbole
\usepackage{manfnt}%			für das Achtung-Symbol
\usepackage{marginnote}
\reversemarginpar

%%%%% Auslassungspunkte %%%%% dots %%%%%
\let\ldotsOld\ldots
\renewcommand{\ldots}{\ldotsOld\unkern}

%%%%% Zwischenraum vor und nach \slash %%%%% Spacing around \slash %%%%%
\let\slashOld\slash
\renewcommand{\slash}{\kern.05em\slashOld\kern.05em}

%%%%% Anführungszeichen %%%%% quotation marks %%%%%
\usepackage[babel,german=guillemets]{csquotes}

%%%%% Listings %%%%% listings %%%%%
\usepackage{listings}
\lstset{%
	frame=none,
%	backgroundcolor=\color{pyellow},
	language=[LaTeX]TeX,
	basicstyle=\ttfamily\small,
	commentstyle=\color{red},
	keywordstyle=, % LaTeX-Befehle werden nicht fett dargestellt
	numbers=none,%left/right
%	numberstyle=\tiny\lnstyle,
%	numbersep=5pt,
%	numberblanklines=false,
	breaklines=true,
%	caption=\lstname,
	xleftmargin=5pt,
	xrightmargin=5pt,
	escapeinside={(*}{*)},
	belowskip=\medskipamount,
	prebreak=\mbox{$\hookleftarrow$}% übernommen vom scrguide (KOMA-Script)
}

%%%%% biblatex %%%%% biblatex %%%%%
\usepackage[%
  bibencoding=utf8,
  style=authortitle-dw,
%  bernhard=true,%
  journalnumber=date
]{biblatex}
\addbibresource{examples/examples-dw.bib}

%%%%% Hyperref %%%%% Hyperref %%%%%
\usepackage{hyperref}
\hypersetup{%
	colorlinks=true,%
	linkcolor=dkblue,% Links
	citecolor=dkblue,% Links zu Literaturangaben
	urlcolor=dkblue,% Links ins Internet
	pdftitle={biblatex-dw},%
	pdfsubject={Dokumentation des LaTeX-Pakets biblatex-dw},%
	pdfauthor={Dominik Waßenhoven},%
	pdfstartview=FitH,%
	bookmarksopen=true,%
	bookmarksopenlevel=2,%
	pdfprintscaling=None,%
}

%%%%% Kurzbefehle %%%%% shortening commands %%%%%
\newcommand{\cmd}[1]{\texttt{\textbackslash #1}}
\newcommand{\option}[1]{\textcolor{dkblue}{#1}}
\newcommand{\wert}[1]{\textcolor{dkblue}{\enquote*{#1}}}
\newcommand{\paket}[1]{\textsf{#1}}
\newcommand{\xbx}[1]{\enquote{#1}}

\makeatletter
\newcommand{\tmp@beschreibung}{}
\DeclareRobustCommand\beschreibung[2][]{% 
  \bgroup 
    \def\tmp@beschreibung{#1}% 
    \ifx\tmp@beschreibung\@empty 
      % leerer Fall 
      \label{#2}%
	    \marginpar{\footnotesize\sffamily\textcolor{dkblue}{#2}}%
    \else 
      % unleerer Fall 
      \label{#1}%
	    \marginpar{\footnotesize\sffamily\textcolor{dkblue}{#2}}%
    \fi 
  \egroup 
}
\newcommand{\tmp@beschreibungcmd}{}
\DeclareRobustCommand\beschreibungcmd[2][]{% 
  \bgroup 
    \def\tmp@beschreibungcmd{#1}% 
    \ifx\tmp@beschreibungcmd\@empty 
      % leerer Fall 
      \label{#2}%
	    \marginpar{\footnotesize\sffamily\textcolor{dkblue}{\textbackslash #2}}%
    \else 
      % unleerer Fall 
      \label{#1}%
	    \marginpar{\footnotesize\sffamily\textcolor{dkblue}{\textbackslash #2}}%
    \fi 
  \egroup 
}
\makeatother

\newcommand{\bl}{\paket{biblatex}}
\newcommand{\bldw}{\paket{biblatex-dw}}

% \optlist[nur bei xy]{Option}{Wert} -> Option (Wert)
% \optset[nur bei xy]{Option}{Wert} -> Option=Wert
% \opt{Option}
% \optnur[nur bei xy]{Option}
% \befehl{Befehlsname}{Definition}{Beschreibung}
% \befehlleer{Befehlsname}{Beschreibung}
\newcommand{\optlist}[3][]{\item[\option{#2}](#3) \emph{#1}\hfill%
  {\footnotesize\sffamily\seite{\pageref{#2}}\\}}
\newcommand{\optset}[3][]{\item[\option{#2=#3}] \emph{#1}\\}
\newcommand{\opt}[1]{\item[\option{#1}]~\hfill%
  {\footnotesize\sffamily\seite{\pageref{#1}}\\}}
\newcommand{\optnur}[2][]{\item[\option{#2}]\emph{#1}\hfill%
  {\footnotesize\sffamily\seite{\pageref{#2}}\\}}  
\newcommand{\befehl}[3]{\item[\option{\cmd{#1}}]\texttt{#2}\\{#3}}
\newcommand{\befehlmitverweis}[2]{\item[\option{\cmd{#1}}]~\hfill%
  {\footnotesize\sffamily\seite{\pageref{#1}}}\\{#2}}
\newcommand{\befehlleer}[2]{\item[\option{\cmd{#1}}]\emph{(\iflanguage{english}{empty}{leer})}\\{#2}}
\newcommand{\biblstring}[3]{%
  \item[\option{#1}]\texttt{#2}\hspace{0.4em}\textbullet\hspace{0.4em}\texttt{#3}}
\newcommand{\feldformat}[3]{\item[{\option{\cmd{DeclareFieldFormat\{{#1}\}\{}\cmd{#2\}}}}]~\\#3}
\newcommand{\eintragstyp}[3]{\item[{\texttt{@#1}}]#2 {\small(in \bl{}:
  \iflanguage{ngerman}{wie}{same as} \texttt{@#3})}\hfill%
  {\footnotesize\sffamily\seite{\pageref{@#1}}}}

\newcommand{\achtung}{\marginnote{\footnotesize\dbend}}

\newcounter{beispiel}
\newcommand{\beispiel}{%
  \iflanguage{english}{Example}{Beispiel}}
\newcommand{\seite}[1]{%
  \iflanguage{english}{page~#1}{Seite~#1}}  

\newcommand{\Mindestanforderung}{%
  \iflanguage{english}%
	  {\textcolor{dkred}{needs at least version~\mindestanforderung{} of \bl{}}}%
		{\textcolor{dkred}{benötigt mindestens Version~\mindestanforderung{} von \bl{}}}}

\newcommand{\Testversion}{%
  \iflanguage{english}%
	  {~and was tested with \bl{} version~\testversion{} and \paket{biber} version~\biberversion{}}%
		{~und wurde mit Version~\testversion{} von \bl{} sowie Version~\biberversion{} von \paket{biber} getestet}}

%%%%% Titelei %%%%% title page %%%%%
\author{Dominik Waßenhoven}
\title{biblatex-dw}
\date{Version~\biblatexdwversion, \biblatexdwdate}

%%%%% Worttrennungen %%%%% Hyphenation %%%%%
\usepackage[htt]{hyphenat}
\hyphenation{
  Stan-dard-ein-stel-lun-gen
}

\nonfrenchspacing%	preamble
%%!TEX encoding = UTF-8 Unicode
% biblatex-dw 
% Copyright (c) Dominik Waßenhoven <domwass(at)web.de>, 2016
%
% This file configures the documentation 
% of biblatex-dw to be printed

%%%%% KOMA-Script Optionen %%%%% KOMA-Script options %%%%%
\KOMAoptions{
	paper=a4
	,headinclude=true
%	,fontsize=12pt,DIV=calc
	,fontsize=11pt,DIV=9
}

%%%%% Mindestanforderung %%%%% Minimum requirement
\renewcommand{\Mindestanforderung}{%
  \iflanguage{english}%
	  {needs at least version~\mindestanforderung{} of \bl{}}%
		{mindestens Version~\mindestanforderung{} von \bl{}}}
		
%%%%% Kopf- und Fußzeilen %%%%% page header and footer %%%%%
\ihead{%
  \iflanguage{english}%
    {biblatex-dw documentation}%
    {Dokumentation von biblatex-dw}}
\ohead{\versionname~\biblatexdwversion, \biblatexdwdate}% Kopf rechts
\cfoot{\pagemark}

%%%%% Lizenzhinweis %%%%% note on licence %%%%%
\renewcommand{\lizenz}{%
  \vfill
  \begin{quote}
  \small
  \itshape
  \iflanguage{english}{%
    This manual is part of the \bldw\ package. It may be distributed and\slash
    or modified under the conditions of the \enquote{\LaTeX\ Project Public
    License}. For more details, please have a look at the
    \enquote{\texttt{README}} file.
   }{%
    Dieses Handbuch ist Teil des Pakets \bldw. Es darf nach den Bedingungen
    der \enquote{\LaTeX\ Project Public Licence} verteilt und\slash oder
    verändert werden. Für weitere Informationen schauen Sie bitte in die Datei
    \enquote{\texttt{LIESMICH}}.
   }%
   \end{quote}
   \clearpage}

%%%%% Inhaltsverzeichnis %%%%% Table of Contents %%%%%
\let\TOC\tableofcontents

%%%%% Keine farbigen Links %%%%% No coloured links %%%%%
\hypersetup{%
	colorlinks=false,%
	%linkcolor=dkblue,% Links
	%citecolor=dkblue,% Links zu Literaturangaben
	%urlcolor=dkblue,% Links ins Internet
	%pdftitle={biblatex-dw},%
	%pdfsubject={Dokumentation des LaTeX-Pakets biblatex-dw},%
	%pdfauthor={Dominik Waßenhoven},%
	%pdfstartview=FitH,%
%	bookmarksopen=true,%
%	bookmarksopenlevel=0,%
	%pdfprintscaling=None,%
}
%		printable version
%!TEX encoding = UTF-8 Unicode
% biblatex-dw 
% Copyright (c) Dominik Waßenhoven <domwass(at)web.de>, 2016
%
% This file configures the documentation 
% of biblatex-dw to be viewed on a screen

%%%%% Satzspiegel %%%%% type area %%%%%
% übernommen aus dem scrguide von Markus Kohm
% adopted from Markus Kohm's scrguide
\usepackage{geometry}
\geometry{papersize={140mm,210mm},%
      includehead,includemp,reversemp,marginparwidth=2em,%
      vmargin={2mm,4mm},hmargin=2mm}%
      
%%%%% Kopf- und Fußzeilen %%%%% page header and footer %%%%%
% zu großen Teilen übernommen aus dem scrguide von Markus Kohm
% for the most part adopted from Markus Kohm's scrguide
\makeatletter
    \setlength{\@tempdimc}{\oddsidemargin}%
    \addtolength{\@tempdimc}{1in}%
    \setheadwidth[-\@tempdimc]{paper}%
\makeatother
\ohead{\smash{%
  \rule[-\dp\strutbox]{0pt}{\headheight}\pagemark\hspace{2mm}}}%
\ihead[%
  {\smash{\colorbox{pyellow}{%
    \makebox[\dimexpr\linewidth-2\fboxsep\relax][l]{%
      \rule[-\dp\strutbox]{0pt}{\headheight}%
      \makebox[2em][r]%
        biblatex-dw, \versionname~\biblatexdwversion, \biblatexdwdate}}}}]%
  {\smash{\colorbox{pyellow}{%
     \makebox[\dimexpr\linewidth-2\fboxsep\relax][l]{%
       \rule[-\dp\strutbox]{0pt}{\headheight}%
       \makebox[2em][r]%
         biblatex-dw, \versionname~\biblatexdwversion, \biblatexdwdate}}}}%

%%%%% Hinweis zur Bildschirmversion %%%%% note on screen version %%%%%
\renewcommand{\screenversion}{%
  \begin{quote}
  \small
  \color{dkred}
  \sffamily
  \iflanguage{english}{%
    This is the screen version of the \bldw{} documentation.
    If you would like to have a printable version, please have a look
    at the \enquote{\texttt{README}} file.
   }{%
    Dies ist die Bildschirmversion der Dokumentation von \bldw{}.
    Wenn Sie eine Druckversion haben möchten, schauen Sie bitte in die
    Datei \enquote{\texttt{LIESMICH}}.
   }%
   \end{quote}}
    
%%%%% Lizenzhinweis %%%%% note on licence %%%%%
\renewcommand{\lizenz}{%
  \vfill
  \begin{quote}
  \small
  \itshape
  \iflanguage{english}{%
    This manual is part of the \bldw\ package. It may be distributed and\slash
    or modified under the conditions of the \enquote{\LaTeX\ Project Public
    License}. For more details, please have a look at the
    \enquote{\texttt{README}} file.
   }{%
    Dieses Handbuch ist Teil des Pakets \bldw. Es darf nach den Bedingungen
    der \enquote{\LaTeX\ Project Public Licence} verteilt und\slash oder
    verändert werden. Für weitere Informationen schauen Sie bitte in die Datei
    \enquote{\texttt{LIESMICH}}.
   }%
   \end{quote}}
     
%%%%% Kein Inhaltsverzeichnis %%%%% No Table of Contents %%%%%
\renewcommand{\TOC}{\clearpage}%		screen version

%%%%% %%%%% %%%%% %%%%% %%%%%
%%%%% Begin of Document %%%%%
%%%%% %%%%% %%%%% %%%%% %%%%%
\begin{document}
\maketitle
\thispagestyle{empty}

\abstract{\noindent \bldw{} is a small collection of styles for the 
 \bl{} package. It was designed for citations in the 
 Humanities and offers some features that are not provided 
 by the standard \bl{} styles. \bldw{} is dependent 
 on \bl{}~-- version~\biblatexdwversion{} \Mindestanforderung{}\Testversion.
 Please note also the requirements of the \bl{} package itself.}

\lizenz
\screenversion
\TOC

\section{Introduction}
\subsection{Installation}
\bldw{} is part of the distributions MiK\TeX{}\footnote{Website: \url{http://www.miktex.org}.} 
and \TeX{}~Live\footnote{Website: \url{http://www.tug.org/texlive}.}~-- thus, you
can easily install it using the respective package manager. If you would like to
install \bldw{} manually, do the following:
Extract the archive \texttt{biblatex-dw.tds.zip} to the \texttt{\$LOCALTEXMF} directory of
 your system.\footnote{If you don't know what that is, have a look at
\url{http://www.tex.ac.uk/cgi-bin/texfaq2html?label=tds} or 
\url{http://mirror.ctan.org/tds/tds.html}.} Refresh your filename database. 
Here is some additional information from the UK \TeX\ FAQ:
\begin{itemize}
	\item \href{%
    http://www.tex.ac.uk/cgi-bin/texfaq2html?label=install-where}{%
    Where to install packages}
	\item \href{%
	  http://www.tex.ac.uk/cgi-bin/texfaq2html?label=inst-wlcf}{%
	  Installing files \enquote{where \LaTeXTeX\ can find them}}
	\item \href{%
	  http://www.tex.ac.uk/cgi-bin/texfaq2html?label=privinst}{%
	  \enquote{Private} installations of files}
\end{itemize}

\subsection{Usage}
The styles are loaded in the same way as the \bl{} standard styles:
 
\begin{lstlisting} 
\usepackage[style=authortitle-dw]{biblatex}
\end{lstlisting}
or
\begin{lstlisting} 
\usepackage[style=footnote-dw]{biblatex}
\end{lstlisting}
The styles are built in a very entangled way which means that the combination of a \bldw\ style with another style is not possible without fail.
For an overview of the styles see the examples \enquote{en-authortitle-dw}
 and \enquote{en-footnote-dw} in the \texttt{examples} folder.

\subsection{Global options and entry options}
The options provided by \bl{} are also available with \bldw{}.
The additional options provided by \bldw{} are described on the next pages. 
There is a general difference between global options and entry options:
global options are valid for all references of a document; they are set
either as optional arguments when loading \bl{} or in a separate config
file (\texttt{biblatex.cfg}). Entry options are set in the field
\texttt{options} of an entry in the bib file. Entry options can sometimes override
global options for the respective entry.

\subsection{Frequently asked Questions (FAQ)}
I answered some frequently asked questions concerning \bl{} and \bldw{} and made them
available online:\\
\url{http://biblatex.dominik-wassenhoven.de/faq.shtml?en}

\subsection{Development}
\bldw{} is an open source project hosted at \href{http://sourceforge.net}{sourceforge.net}. The code (also of the latest, not released version) can be downloaded.\footnote{\url{http://sourceforge.net/p/biblatex-dw/code}.} At sourceforge.net you have also the possibility to file bug reports (if possible, including a minimal example)\footnote{\url{http://sourceforge.net/p/biblatex-dw/tickets/milestone/Bugs}.} and feature requests.\footnote{\url{http://sourceforge.net/p/biblatex-dw/tickets/milestone/Features}.}

\section{The \xbx{authortitle-dw} style}
This style is based on the standard \xbx{authortitle} style.
Besides some changes in punctuation, there are the following differences:

\subsection{Appearance in the bibliography}
\begin{itemize}
	\item The\beschreibung{namefont} font shape of authors and editors can be set by the options 
	      \option{namefont} and\beschreibung{firstnamefont} \option{firstnamefont} which can take the values 
	      \wert{smallcaps}, \wert{italic}, \wert{bold} and \wert{normal}. 
	      If you set \option{useprefix=true}, \option{namefont} affects also 
	      the name prefix (i.\,e. \enquote{von}, \enquote{de} etc.). If \option{%
	      useprefix=false} is set (which is the default), the name prefix depends
	      on the option \option{firstnamefont} which in every case affects the
	      name suffix (the \enquote{junior} part).
	\item If\beschreibung{oldauthor} you set the \option{namefont}, but nevertheless need some
	      of the names being typeset in upright shape (e.g. medieval or
	      antique authors), you can add \texttt{options\,=\,\{oldauthor=true\}} 
	      to the respective entry of your
	      bib file. If you have to switch back to the normal appearance also
	      for entries with this \texttt{oldauthor} flag, you can set the global
	      option \option{oldauthor=false} in order to override the entry option.
  \item There\beschreibung{oldbookauthor} is also the entry option \option{oldbookauthor}
        which is the same as \option{oldauthor} but for the bookauthor. This is 
        useful for \texttt{@inbook} entries representing, e.g., an introduction to an 
        edition of a work by an author that should not be typeset in the usual font 
        for last names. This option can be set on a per entry basis and can be 
        disabled with the global option \option{oldauthor=false}.
	\item The\beschreibung{idemfont} font shape of the \enquote{idem} string (see below) can be set by
	      the option \option{idemfont} which can take the values 
	      \wert{smallcaps}, \wert{italic}, \wert{bold} and \wert{normal}. If you
	      do not use this option, the \enquote{idem} string is printed in the 
	      same font shape as indicated by the option \option{namefont}.
	\item The\beschreibung{ibidemfont} font shape of the \enquote{ibidem} string (see below) can be set by
	      the option \option{ibidemfont} which can take the values 
	      \wert{smallcaps}, \wert{italic}, \wert{bold} and \wert{normal}. The 
	      default value is \wert{normal}.
  \item The\beschreibung{acronyms}\beschreibung{acronym}
	      \texttt{shorthands} and journal abbreviations (\texttt{shortjournal})
        can be set with the command \cmd{mkbibacro} (default for this command:
        \textsc{smallcaps}). For that, you need to set the global option
        \option{acronyms} to \wert{true} \emph{and} the entry option \option{acronym=true}.
        If you want to customize the command \cmd{mkbibacro}, see section
        \enquote{\nameref{mkbibacro-anpassen}} on page~\pageref{mkbibacro-anpassen}.
	\item The\beschreibung{idembib} option \option{idembib} provides a possibility to substitute
	      identical authors\slash editors in subsequent entries in the
	      bibliography by an idem phrase. If set to \wert{false}, the names are
	      given also in subsequent entries of the same authors\slash editors.
	      Using \option{idembib=true}, the substitution is enabled. The format
	      is then set by the option\beschreibung{idembibformat} \option{idembibformat} which can take the
	      values \wert{idem} to get \enquote{idem} instead of the name(s) and 
	      \wert{dash} to get a dash (---). In some languages, the idem phrase
	      is gender-specific. The gender for authors\slash editors has to be
	      given in the \texttt{gender} field of your bib file (see the \bl{}
	      documentation for details). The default value for \option{idembib} is
	      \wert{true}, the default value for \option{idembibformat} is
	      \wert{idem}.
	\item When\beschreibung{edbyidem} \texttt{author} and \texttt{editor} are the same in 
	      \texttt{@inbook}, \texttt{@incollection} or \texttt{@inreference} entries, the name is not 
	      repeated but substituted by the string \enquote{idem}. This feature
	      is controled by the option \option{edbyidem} which can be set to 
	      \wert{true} or \wert{false}; the default value is \wert{true}.
  \item The\beschreibung{editorstring} option \option{editorstring} can take the values 
	      \wert{parens}, \wert{brackets} and \wert{normal}; the default is
	      \wert{parens}. This option sets the string \enquote{editor}
	      (abbreviated \enquote{ed.}) in parentheses or in brackets. If you set 
	      the option to \wert{normal}, the editor string is not 
	      surrounded by parentheses or brackets. Instead, a comma is added after 
	      the editor's name.
	      If \option{usetranslator=true} is used, the setting for
	      \option{editorstring} is also valid for the string
	      \enquote{translator} (abbreviated \enquote{trans.}).
	\item The\beschreibung{editorstringfont} option \option{editorstringfont}
        determines the font used for the editor string (and 
        translator string). With \wert{normal}, 
        the normal font is used, with \wert{namefont}, the setting for the option 
        \option{namefont} is also used for the editor string. The default value for this 
        option is \wert{normal}.
  \item Using\beschreibung{pseudoauthor} the entry option \option{pseudoauthor}, the author can be put 
	      in brackets or omitted. This is useful for editions of works whose authors 
				are not named, but are known, for instance. If the global option \option{pseudoauthor}
				is set to \wert{true} (and the entry option \option{pseudoauthor} is used), the author of this 
				entry is printed. The new commands \cmd{bibleftpseudo} and \cmd{bibrightpseudo} 
				are used to enclose the author. These commands are empty by default. If you would like to enclose
				the author by brackets, for instance, you have to redefine the commands:
				\begin{lstlisting}
\renewcommand*{\bibleftpseudo}{\bibleftbracket}
\renewcommand*{\bibrightpseudo}{\bibrightbracket}
				\end{lstlisting}
				If the global option 
				\option{pseudoauthor} is set to \wert{false}, the author of entries 
				with the entry option \option{pseudoauthor} are not printed at 
				all. The default value for the global option is \wert{true}
				(i.e. the behaviour is the same regardless of the entry 
				option \option{pseudoauthor}).
  \item With\beschreibung{nopublisher} default settings, the \texttt{publisher} is not printed, only 
        \texttt{location} and \texttt{date}. If you would like to have the
        publisher printed, you have to set the option 
        \option{nopublisher=false}.
  \item You\beschreibung{nolocation} can also suppress the location with \option{nolocation=true}. In 
        this case, also the publisher is omitted (regardless of the setting of
        \option{nopublisher}). The default setting is \wert{false}.
  \item The\beschreibung{pagetotal} fields \texttt{doi}, \texttt{eprint}, \texttt{isbn}, \texttt{isrn},
        \texttt{issn} and \texttt{pagetotal} are not printed with the default 
        settings. They can however be switched on with the options 
        \option{doi=true}, \option{eprint=true}, \option{isbn=true} (which is also valid
				for the fields \texttt{isrn} and \texttt{issn}) or \option{pagetotal=true},
        respectively.       
  \item The\beschreibung{origfields} option \option{origfields} lets you decide, whether you would like 
        to have the fields \texttt{origlocation}, \texttt{origpublisher} and 
        \texttt{origdate} printed or not; the standard is \wert{true}. If you 
        use the option and the field \texttt{origlocation} is set, the 
        \enquote{orig} fields will be printed. In this case, the fields 
        \texttt{location}, \texttt{publisher} and \texttt{date} are appended 
        in parentheses, preluded by the bibstring \texttt{reprint}. Note that 
        the fields \texttt{publisher} and \texttt{origpublisher} are only 
        printed, if the option \option{nopublisher=false} is given.
        \achtung Note also that the \texttt{edition} field applies to the 
        original edition, as reprints are normally not released in more than 
        one edition, but re-issue a specific edition. If the option 
        \option{edsuper} is used, the edition is printed as superscript number 
        ahead of \texttt{origdate}.
  \item With\beschreibung{origfieldsformat} the option \option{origfieldsformat}, which can take the values 
        \wert{parens}, \wert{brackets} and \wert{punct}, you can set the 
        appearance of the reprint details (with \option{origfields=true}). The 
        values \wert{parens} and \wert{brackets} put them in parentheses or 
        brackets, respectively. The default is \wert{punct}; this means that 
        the reprint details are introduced by the punctuation command 
        \cmd{origfieldspunct}, which is preset to a comma.
  \item The punctuation before \texttt{titleaddon}, \texttt{booktitleaddon} 
        and \texttt{maintitleaddon} is controlled by the new command 
        \cmd{titleaddonpunct}. The default is a period.
	\item With\beschreibung{edsuper} \option{edsuper=true},
	      the edition is printed as superscript number (not as ordinal number) 
	      just before the year. The default value for this option is 
	      \wert{false}.\\
        \achtung Note that this works only, if you have integers in the field
        \texttt{edition}, and nothing but integers. Information like 
        \enquote{5th, revised and expanded edition} in the \texttt{edition} 
        field will be printed as usual, not as a superscript number.
        Additionally, a warning will appear. If you would like to use the 
        option \option{edsuper} for one of your documents, you should ensure 
        that you put only integers into the \texttt{edition} field and use the 
        field \texttt{note} for more detailed information on editions.
	\item With\beschreibung{editionstring} \option{editionstring=true}, the 
        bibliography string \enquote{edition} will be added to the 
        \texttt{edition} field, even if it is not an integer. Thus, 
        you can type, e.g., 
	      \begin{lstlisting}
edition = {2., revised}
	      \end{lstlisting}
        in your bib file and you will get \enquote{2., revised ed.}
        \option{editionstring=false} will give the additional bibliography
        string only if there is an integer in the \texttt{edition} field
        (which is \bl's standard behaviour).
        The default for this option is \wert{false}. 
	\item If\beschreibung{shortjournal} the option \option{shortjournal} is set to \wert{true}, the
	      field \texttt{shortjournal} is used instead of \texttt{journaltitle}.
	      This is useful for journal abbreviations.
  \item If the \texttt{volume} field is not present for a journal, the 
        \texttt{year} is \emph{not} printed in parenthesis: \enquote{Journal 
        name 2008}. But if the \texttt{month} field is set (or the \texttt{date} field 
				contains a month, e.g. \texttt{2008-03}), the date is
        separated from the journal title by an additional comma.
	\item The\beschreibung{journalnumber} option \option{journalnumber} allows you to adjust the position of
	      a journal's \texttt{number}: with \wert{standard}, the behaviour of the
	      standard styles is used, but you can configure the separator between
	      \texttt{volume} and \texttt{number} with the new command
	      \cmd{jourvolnumsep} (default: \cmd{adddot}). With \wert{afteryear} the
	      number is printed after the \texttt{year} and introduced by the
	      command \cmd{journumstring}: \enquote{Journal name 28 (2008), no.~2}.
	      The value \wert{date} assures that the date is printed, even if the 
	      field \texttt{issue} is given (this is not the case in the standard
	      styles).
	      Additionally, \option{journalnumber=date} prints the number before the
	      date, if the date (at least year and month) is given, but it prints
	      the number after the year, if only the year is given (i.e. if the date
	      contains only a year). See section~\ref{journalnumberdate} on
	      page~\pageref{journalnumberdate} for details. The default for
	      \option{journalnumber} is \wert{standard}.	      
	\item The command \cmd{journumstring} introduces the journal number.
	      The standard is \wert{, no.~}. The command can be redefined,
	      e.g.:
	      \begin{lstlisting}
\renewcommand*{\journumstring}{\addspace}
				\end{lstlisting}
	\item The command \cmd{jourvolstring} introduces the journal volume.
	      The standard is a space. The command can be redefined, e.g.:
				\begin{lstlisting}
\renewcommand*{\jourvolstring}{%
  \addspace vol\adddot\space}
				\end{lstlisting}
  \item The\beschreibung{series} option \option{series} affects the position of the \texttt{series}
        field, possible values are \wert{afteryear}, \wert{beforeedition} and \wert{standard}. This affects the entry types \texttt{@book}, \texttt{@inbook},
        \texttt{@collection}, \texttt{@incollection}, \texttt{@proceedings},
        \texttt{@inproceedings} and \texttt{@manual}. 
        If set to \wert{afteryear} or \wert{beforeedition}, the series will be put after the year or before the edition.
        The standard value for this option is \wert{standard}, the series will be printed before the location.
	\item The\beschreibung{seriesformat} option \option{seriesformat} can 
        take the values \wert{standard} and \wert{parens}. If set to 
        \wert{parens}, \texttt{series} and \texttt{number} (of a book etc.)
        will be printed in parentheses, otherwise without (as is the biblatex 
        standard behaviour). The default value for this option is \wert{parens}.
  \item The command \cmd{seriespunct} is the punctuation before the 
        \texttt{series} (of a book etc.). With \option{seriesformat=parens}, 
        this punctuation is set inside the
        parentheses. You can set it e.g. to \wert{=\cmd{addspace}} which is
        common in some fields. The default is empty, i.e.\ the series is 
        printed right after the opening round bracket.
  \item The command \cmd{sernumstring} is the punctuation between the 
        \texttt{series} (of a book etc.) and its \texttt{number}. The default 
        is \wert{\cmd{addspace}}. You can easily redefine it, e.g.:
				\begin{lstlisting}
\renewcommand*{\sernumstring}{%
  \addcomma\space\bibstring{volume}\addspace}
				\end{lstlisting}
        This would give \enquote{(Name of Series, vol. N)}.
  \item Using\beschreibung{shorthandinbib} option \option{shorthandinbib=true},
	      the shorthands are printed in the bibliography, enclosed by brackets, preceding 
				the respective entry. The appearance is controlled by the field format 
				\enquote{shorthandinbib} which can of course be redefined. If you want to get 
				rid of the default brackets, for instance, you should do the following:
				\begin{lstlisting}
\DeclareFieldFormat{shorthandinbib}{#1}
				\end{lstlisting}
				In this case there will only be a space between shorthand and bibliography entry,
				so that you should redefine the punctuation as well which is printed by the
				command \cmd{shorthandinbibpunct}. You could alter it to print, for example, an equal 
				sign, but you should use \cmd{nopunct} in order to avoid superfluous punctuation:
				\begin{lstlisting}
\renewcommand*{\shorthandinbibpunct}{%
  \addspace=\nopunct\addspace}
				\end{lstlisting}
  \item The\beschreibung{annotation} option \option{annotation} is introduced. If it is set to
	      \wert{true}, the field \texttt{annotation} will be printed in 
	      \textit{\small small italic} typeface at the end of the entry. The
	      default value for this option is \wert{false}.
	      You can alter the appearance by redefining the following command:
				\begin{lstlisting}
\renewcommand{\annotationfont}{\small\itshape}
				\end{lstlisting}
	\item The\beschreibung{library} option \option{library} is introduced. If it is set to
	      \wert{true}, the field \texttt{library} will be printed in 
	      {\small\sffamily small sans serif} typeface at the end of the entry. 
	      The default value for this option is \wert{false}.
	      You can alter the appearance by redefining the following command:
				\begin{lstlisting}
\renewcommand{\libraryfont}{\small\sffamily}
				\end{lstlisting}
	\item If both \option{annotation} and \option{library} are set to 
	      \wert{true}, the contents of the \texttt{annotation} field is placed 
	      before the contents of the \texttt{library} field.
	\item In\beschreibung{@inreference}
        normal \bl{}, the entry type \texttt{@inreference} is an alias for 
	      \texttt{@incollection} entries. In \bldw{}, this entry type can be used 
	      for articles in encyclopedias or dictionaries. The output is similar to 
	      that of an \texttt{@incollection}, but there are some differences:
	      \begin{itemize}
          \item The title is enclosed in quotes.
          \item The bibliography string \enquote{inrefstring} (standard:
                \enquote{article}\slash\enquote{art.}) is put in front
                of the title.
          \item The location is not printed.
          \item With a \texttt{volume} present, the output is e.g. \enquote{in:
                Encyclopedia 2 (1990), p. 120.}
        \end{itemize}
        The \texttt{@reference} entry type still is an alias to \texttt{@collection}.
        Thus, you can use either \texttt{@collection} or \texttt{@reference} for
        a work of reference.
  \item Using the\beschreibung{inreference} option \option{inreference=full},
  			\texttt{@inreference} entries are printed always in full, but are omitted 
				from the bibliography. Note that when using \option{xref=false}, you have to 
				cite the corresponding \texttt{@reference} entry manually (e.g. with \cmd{nocite}) 
				if you wish to include the \texttt{@reference} entry in the bibliography. The 
				\option{inreference} option complies with both \option{ibidtracker} and
			  \option{idemtracker}. The default for this option is \wert{normal}.

	\item In\beschreibung{@review}
        normal \bl{}, the entry type \texttt{@review} is an alias for 
	      \texttt{@article} entries. In \bldw{}, this entry type can be used 
        for reviews. The output is similar to \texttt{@article}, with the
        following differences:
        \begin{itemize}
          \item The title is enclosed in quotes.
          \item The bibliography string \enquote{reviewof} (standard:
                \enquote{Review of}) is put in front of the title.
          \item Instead of manually entering the data for the reviewed work in the \texttt{title}
                field of the \texttt{@review} entry, it is also possible to add a reference in the field
                \texttt{xref}. The entry with the \BibTeX\ key given in the \texttt{xref} field
                will be cited. Using this approach, options like \option{namefont} or 
                \option{firstfull} are considered also for the reviewed work.
        \end{itemize}
\end{itemize}

\subsection{Appearance in citations}
\begin{itemize}
	\item In citations of the same author\slash editor as the
	      immediately preceding citation the name is replaced by 
	      the string \enquote{idem} unless the citation is the first 
	      one on the current page. This feature is controled by the
	      \bl{} option \option{idemtracker} which is set to \wert{constrict}. If
	      you would like to switch off the idem functionality, you can use 
	      \option{idemtracker=false}. For more information on the
	      \option{idemtracker} option see the \bl{} manual.
	\item Immediately repeated citations are replaced by the string
	      \enquote{ibidem} unless the citation is the first one on
	      the current page. This behaviour can be suppressed with
	      the \bl{} option \option{ibidtracker=false}. In this case,
	      the \enquote{idem} functionality
	      is still active as long as you do not set the option
	      \option{idemtracker=false}.
	\item The\beschreibung{shorthandibid} option \option{shorthandibid} controls whether
	      immediately repeated citations with a \texttt{shorthand} 
	      should also be replaced by the string \enquote{ibidem} or 
	      not. Possible values are \wert{true} and \wert{false}, 
	      the default value is \wert{true}. Note that this option has
	      no effect if you set the option \option{ibidtracker=false}.
	      Note also that this behaviour can be overridden
	      for each entry by using \texttt{options\,=\,\{shorthandibid=true\}} or 
	      \texttt{options\,=\,\{shorthandibid=false\}}, respectively.
  \item With\beschreibung{addyear} the option \option{addyear}, the year of the 
			  publication will be added after the title. The year appears in 
		    parentheses. The default for this option is \wert{false}.
	\item With\beschreibung{omiteditor} \option{useeditor=false} and \option{omiteditor=true}, the editor 
	      in citations will be omitted. The same applies to the short citations in
				the bibliography, if the xref functionality is in use. With 
				\option{useeditor=true}, the option \option{omiteditor} has
				no effect. The default for this option is \wert{false}.
	\item With\beschreibung[edstringincitations]{edstring\-incitations}
	      the option \option{edstringincitations=true}, the editor and 
	      translator strings are shown in citations (not only in the full 
	      citations). The appearance depends on how the option \option{editorstring}
	      is set. The default value for this option is \wert{true}.
	\item If you use the \cmd{textcite} command with an entry that
	      lacks author and editor, \bl{} will give you a warning
	      and output the entry's key in bold face.
	\item The\beschreibung{firstfull} option \option{firstfull} delivers a full citation for the first
	      occurence of an entry. The default value for this option is 
	      \wert{false}.
	\item If\beschreibung{citedas} a \texttt{shorthand} is given and the option \option{firstfull}
	      is set to \wert{true}, the addition \enquote{henceforth 
	      cited as~\ldots} will be added to the citation. This behaviour can
	      be influenced by the option \option{citedas}, which can take
	      the values \wert{true} or \wert{false}; the default
	      is \wert{true}. Note that this behaviour can also be overridden
	      for each entry by using \texttt{options\,= \{citedas=true\}} or 
	      \texttt{options\,=\,\{citedas=false\}}, respectively.
  \item The\beschreibung{citepages}
	      option \option{citepages} allows you to suppress pages or page ranges
				in full citations, if the field \texttt{pages} is given:
				\option{citepages=permit} allows duplicates, i.e. the \texttt{pages}
				field as well as the \texttt{postnote} are printed;
				with \option{citepages=suppress} the \texttt{pages} field of the full citations
				will be suppressed in any case, thus only the \texttt{postnote} is printed;
				\option{citepages=omit} omits the \texttt{pages} only if the \texttt{postnote}
				is a number;
				\option{citepages=separate} always prints the \texttt{pages} field, but separates
				the \texttt{postnote} by using the string \enquote{here}, if it is a number. 
				In this case, the \emph{bibliography string} \enquote{thiscite} is used.
				The standard value for this option is \wert{separate}.
				A small illustrating example which uses the following \cmd{cite} commands:
				\begin{lstlisting}
\cite{key}
\cite[a note]{key}
\cite[125]{key}
				\end{lstlisting}
				%
				\option{citepages=permit}:
				\begin{quote}
				Author: Title, in: Book, pp.\,100--150.

				Author: Title, in: Book, pp.\,100--150, a note.

				Author: Title, in: Book, pp.\,100--150, p.\,125.
				\end{quote}
				%
				\option{citepages=suppress}:
				\begin{quote}
				Author: Title, in: Book.

				Author: Title, in: Book, a note.

				Author: Title, in: Book, p.\,125.
				\end{quote}
				%
				\option{citepages=omit}:
				\begin{quote}
				Author: Title, in: Book, pp.\,100--150.

				Author: Title, in: Book, pp.\,100--150, a note.

				Author: Title, in: Book, p.\,125.
				\end{quote}
				%
				\option{citepages=separate}:
				\begin{quote}
				Author: Title, in: Book, pp.\,100--150.

				Author: Title, in: Book, pp.\,100--150, a note.

				Author: Title, in: Book, pp.\,100--150, here p.\,125.
				\end{quote}
	\item The\beschreibung{citeauthor} \option{citeauthor} option controls the name format for
				the \cmd{citeauthor} command; it can take the values \wert{namefont}, 
				\wert{namefontfoot} and \wert{normalfont}. With \option{citeauthor=namefont} the 
				same format as set by the option \option{namefont} is used. This is the default 
				behaviour. With \option{citeauthor=normalfont} the normal font is always used for 
				\cmd{citeauthor}, regardless of the \option{namefont} option. With 
				\option{citeauthor=namefontfoot} the \option{namefont} format is used if the
	      \cmd{citeauthor} command is inside a footnote, otherwise the normal font is used.
	\item The\beschreibung{citeauthorname} option \option{citeauthorname} controls the
				appearance of names when using the the commands \cmd{citeauthor} and \cmd{textcite}.
				With \option{citeauthorname=firsfull}, the name is given in full when cited for the 
				first time; from the second citation onwards, only the last name is printed. This 
				works also with different citations from the same author, regardless whether you 
				only use \cmd{citeauthor} or \cmd{textcite} or both. With 
        \option{citeauthorname=full} the name is always given in full, with 
        \option{citeauthorname=normal} only the given name is printed.
        The default value for this option is \wert{normal}.
	\item If\beschreibung{singletitle} you use the \bl{} option \option{singletitle=true}, the title of 
	      a cited work is only printed if there is more than one work of the same
				author. See the biblatex documentation for details.
\end{itemize}

\subsection{List of Shorthands}
\begin{itemize}
	\item The\beschreibung{terselos} list of shorthands contains only author (or editor), title and,
	      if necessary, booktitle or maintitle. This information should be
	      sufficient in order to find the full bibliographical data in the
	      bibliography. This feature is controled by the
	      option \option{terselos} which can be set to \wert{true} or
	      \wert{false}; the default value is \wert{true}.
	\item Using\beschreibung{shorthandwidth} the option \option{shorthandwidth}, you can define the width of
	      the labels in the list of shorthands. This is especially useful when
	      you have very long shorthands. The option can take any length 
	      definition for example \wert{40pt} or \wert{3em}. \achtung If you use 
	      the option \option{shorthandwidth}, the spacing after the label is 
	      reduced and a colon is inserted after every label. The punctuation 
	      mark can be redefined with the command \cmd{shorthandpunct} and the 
	      spacing is assigned by the new length \cmd{shorthandsep}. The standard 
	      values (as soon as \option{shorthandwidth} is used) are:
				\begin{lstlisting}
\renewcommand{\shorthandpunct}{\addcolon}
\setlength{\shorthandsep}{3pt plus 0.5pt minus 0.5pt}
				\end{lstlisting}
\end{itemize}

\section{The \xbx{footnote-dw} style}

This style is similar to \xbx{verbose-inote}. It is based on the 
\xbx{authortitle-dw} style. Thus, you can use all options defined in 
\xbx{authortitle-dw}; the only exceptions are the options  \option{addyear},
\option{firstfull} and \option{inreference}.
Apart from that, there are the following differences between \xbx{footnote-dw} and
\xbx{authortitle-dw}:
\medskip 
\begin{itemize}
	\item Citations are \emph{only} possible inside footnotes. Citations
	      which are not inside footnotes will automatically be turned into 
	      a \cmd{footcite}. The only exception is \cmd{textcite} which will
	      give the author's name in the text and a footnote citation; when used inside
	      a footnote, \cmd{textcite} will give the author's name, followed by the
	      citation in parenthesis.
	\item The first citation will give a full reference, following citations
	      will only use \texttt{author} and \texttt{shorttitle} (or 
	      \texttt{title}, if no \texttt{shorttitle} is given) with the
	      addition \enquote{see n.~\enquote{N}}, where \enquote{N} is
	      the number of the footnote where the first citation occured.
	\item The\beschreibung{pageref} option \option{pageref} known from \bl's \xbx{verbose-note} and 
	      \xbx{ver\-bose-inote} styles is also available. When set to \wert{true},
	      the page number is added to the footnote number pointing to the full 
	      citation if it is located on a different page. This option defaults to 
	      \wert{false}.
	\item When using \cmd{parencite} outside a footnote, the parentheses 
	      will be dropped and a \cmd{footcite} will be used instead. Inside
	      footnotes, the \cmd{parencite} command will work as expected. The
	      addition \enquote{see n.~\ldots} will be surrounded by brackets
	      instead of parentheses.
	\item If one or both of the options \option{annotation} and \option{library} 
	      are set to \wert{true}, the annotations and library information
	      are only printed in the bibliography (if there is one), but not in the 
	      first citations and in the list of shorthands.
\end{itemize}

\section{Crossref functionality}
\label{xreffunctionality}
\subsection{How it works}
The\beschreibung{xref} crossref functionality of \bldw{} provides a possibility for
dependent papers to refer to a parent entry. In order to use it, you
have to create a parent entry of the type \texttt{@book}, \texttt{@collection}
or \texttt{@proceedings}. Each child entry belonging to this parent entry can
refer to its \BibTeX\ key using the field \texttt{xref}. This works for entries
of the type \texttt{@inbook}, \texttt{@incollection} and \texttt{@inproceedings}.

Here is a small example:
\begin{lstlisting}
@collection{parent,
  editor = {(*\emph{Editor}*)},
  title = {(*\emph{Book Title}*)},
  location = {(*\emph{Location}*)},
  date = {2008}
}
@incollection{child,
  author = {(*\emph{Author}*)},
  title = {(*\emph{Title of the Contribution}*)},
  xref = {parent}% reference
}
\end{lstlisting}
When an \texttt{@incollection} entry is cited and the option \option{xref} is set
to \wert{true}, the cited entry takes data of the entry with
the \BibTeX\ key \texttt{parent}. If it is available, the \texttt{shorthand}
is used. Otherwise, the fields \texttt{author}\slash\texttt{editor} and \texttt{title}
(or \texttt{shorttitle}, if available) are printed. Thus, the reader is
referred to the corresponding parent entry in the bibliography and in this way
is provided with all relevant data. 

With multiple child entries, it would be possible to have the data entered only
once (and thus, also the possibility to have typos would be reduced). But you
have to keep in mind that you then \emph{always} have to use this crossref 
mechanism. If you need a document with all data in every single entry, the corresponding
information (\texttt{editor}, \texttt{booktitle} etc.) would be missing. It is
therefore better to enter all relevant data for \texttt{@incollection}, \texttt{@inbook} 
and \texttt{@inproceedings} entries including the \BibTeX{} key of the parent
entry in the \texttt{xref} field. 

The behaviour of the \texttt{xref} field is affected by the package option
\option{mincrossrefs}, which has the default setting \wert{2}. That means, if only
one article of a collection is cited, the collection would not be included in the
bibliography (if it is not cited explicitly) and thus the article would be incomplete.
For that reason, the option \option{mincrossrefs} is set to \wert{1} when using
\option{xref=true}.

The default setting does not use this crossref functionality. You can also switch 
it off with \option{xref=false}. In both cases, the \texttt{xref} field takes effect
only insofar as the parent entry is only included in the bibliography if at least
two of his child entries are cited (\option{mincrossrefs=2}). You can of course set
the value for \option{mincrossrefs} to whatever you want or need.

\achtung{}The reference only works with the field \texttt{xref}. The field \texttt{crossref}
does \emph{not} work together with this crossref functionality! Instead of simply 
copying the missing fields from the parent entry into the child entry, as it is
done in standard \BibTeX\ and its crossref method, the crossref functionality
presented here uses a special citation which provides the relevant data of the
parent entry.

\subsection{Peculiarities}
\subsubsection{\xbx{footnote-dw} specifics}
The crossref functionality also works with \xbx{footnote-dw}. When using the option
\option{xref=true}, citing the parent entry for the first time sets a \cmd{label} 
for the parent entry. It does not matter if the parent entry is cited itself or via 
a child entry. When a (second) child entry is cited, a short citation of the parent 
entry is given along with a reference to the footnote of the first citation in which the
parent entry appeared.

\subsubsection{Multi-volume \enquote{parents}}
If a parent enry is a multi-volume work, the child entry normally refers to
a specific volume of the parent entry, not to the entire work. In order to
take that into account, \bldw{} checks if the \texttt{volume} field is given
in the child entry. If so, it checks if the parent entry has a \texttt{volume},
too. Only if the parent entry has no \texttt{volume} specified, the volume of
the child entry is printed immediately before the pages.%
\footnote{The reason why it is not checked if the content of both \texttt{volume}
fields are the same is the following: As soon as the \texttt{volume} field of 
the parent entry is given, it represents a single volume of a multi-volume work.
If we now have a child entry with a different \texttt{volume}, it can be considered
as a fault. At least no scenario came to my mind where a child entry needs a different
volume as the correpsonding parent entry.} In this case, it will also be checked
if the data in the \texttt{date} field match. If they don't match (e.g. when the
multi-volume work was published in several years), the year will be printed
additionally. Moreover, it will finally be checked if the data in the field
\texttt{location} (or \texttt{address}) match. If they don't match (e.g. when the
multi-volume work was published at different places, but the single volume referred to
by the child entry was published only at one place), the location will be printed
immediately before the year.

\section{Survey of options}
\subsection{Global options}
Global options are valid for all references of a document; they are set
either as optional arguments when loading \bl{} or in a separate config
file (\texttt{biblatex.cfg}). The value in parentheses shows the default.

	\optlist{acronyms}{false}
	  Only if set to \wert{true}, the entry option \option{acronym} will be regarded.
	\optlist[\xbx{authortitle-dw} only]{addyear}{false}
	  If set to \wert{true}, the year of the publication will be set after the title
	  in citations, appearing in parentheses.
	\optlist{annotation}{false}
	  The field \texttt{annotation} is printed at the end of the bibliography 
	  item.
	\optlist{citeauthor}{namefont}
	  Specifies the font shape of the authors' names when the \cmd{citeauthor}
	  command is used. Possible values are \wert{namefont}, \wert{normalfont} and 
	  \wert{namefontfoot}.
	\optlist{citeauthorname}{normal}
	  Controls the name output when using \cmd{citeauthor} or \cmd{textcite}. With 
	  \wert{firstfull} the full name is printed at first citation, at all 
	  subsequent citations only the last name is printed. With \wert{full} the full name is 
	  always printed, whereas \wert{normal} always gives the given name only.
	\optlist{citedas}{true}
	  The first citation (in \xbx{authortitle-dw} only if the option 
	  \option{firstfull} is used) is followed by the string
	  \enquote{henceforth cited as} whenever a \texttt{shorthand} is given.
	\optlist{citepages}{separate}
	  Specifies if the pages of a fullcite or of the first citation (\xbx{authortitle-dw}: 
		only with option \option{firstfull}) of an entry with \texttt{pages} field
		will be printed or not.
	\optlist{edbyidem}{true}
	  \enquote{ed. by idem} instead of \enquote{ed. by \emph{Editor}}.
	\optlist{editionstring}{false}
    Adds the string \enquote{ed.} to the edition, regardless of the content of the 
    \texttt{edition} field.
	\optlist{editorstring}{parens}
	  Sets the editor string (with \option{usetranslator=true} also the
	  translator string) in parentheses (\wert{parens}) or brackets 
	  (\wert{brackets}). If set to \wert{normal}, the editor string is
	  put after the editor's name and preceded by a comma.
	\optlist{editorstringfont}{normal}
    The editor\slash translator strings are typeset either in normal font (\wert{normal}) or
    in the font used by \option{namefont} (\wert{namefont}).
	\optlist{edstringincitations}{true}
	  In citations, the editor string (with \option{usetranslator=true}
	  also the translator string) is put after the editor's name (and the
	  translator's name, where appropriate).
	\optlist{edsuper}{false}
	  The edition is printed as superscript number straight ahead of the year.
  \optlist[\xbx{authortitle-dw} only]{firstfull}{false}
    The first citation is printed with full reference.
	\optlist{firstnamefont}{normal}
	  Specifies the font shape of the first names of authors and editors as well 
	  as of name affixes and (if \option{useprefix} is set to \wert{false}) of 
	  name prefixes. Possible values are \wert{smallcaps}, \wert{italic}, 
	  \wert{bold} and \wert{normal} (which is the default and means that the 
	  normal font shape is used).
	\optlist{ibidemfont}{normal}
	  Specifies the font shape of the \enquote{ibidem} string.
	  Possible values are \wert{smallcaps}, \wert{italic}, \wert{bold} and 
	  \wert{normal} (which is the default and means that the normal font shape 
	  is used).
	\optlist{idembib}{true}
	  \enquote{Idem} or \enquote{---} instead of names for the same 
	  authors\slash editors of subsequent entries in the bibliography.
	\optlist{idembibformat}{idem}
	  Only for \option{idembib=true}: With \wert{idem} the names are substituted 
	  by \enquote{Idem}, with \wert{dash} they are substituted by
	  a\,---\,well\,---\,dash (\enquote{---}).
	\optnur[no default]{idemfont}
	  Specifies the font shape of the \enquote{idem} string.
	  Possible values are \wert{smallcaps}, \wert{italic}, \wert{bold} and 
	  \wert{normal}. If this option is not set, the font shape indicated by the 
	  option \option{namefont} is used (this is the default behaviour).
	\optlist[\xbx{authortitle-dw} only]{inreference}{normal}
	  If set to \wert{full}, \texttt{@inreference} entries are printed in full
	  in the citations, but are omitted from the bibliography.
	\optlist{journalnumber}{standard}
	  Position of a journal's \texttt{number}: with \wert{standard} as in the 
	  standard styles, with \wert{afteryear} after the \texttt{year}, introduced 
	  by the bibliography string \enquote{number} (\enquote{no.}), and with 
	  \wert{date} dependent on the date settings (see 
	  section~\ref{journalnumberdate} on page~\pageref{journalnumberdate}).
	\optlist{library}{false}
	  The field \texttt{library} is printed at the end of the bibliography item.
	\optlist{namefont}{normal}
	  Specifies the font shape of the last names of authors and editors as well 
	  as of name prefixes (if \option{useprefix} is set to \wert{true}).
	  Possible values are \wert{smallcaps}, \wert{italic}, \wert{bold} and 
	  \wert{normal} (which is the default and means that the normal font shape 
	  is used).
	\optlist{nopublisher}{true}
	  The publisher is not printed.
	\optlist{nolocation}{false}
	  If set to \wert{true}, the location is not printed. In this case, the 
	  publisher is omitted, too, even if \option{nopublisher} is set to 
	  \wert{false}.
	\optlist{oldauthor}{true}
	  If set to \wert{false}, the entry options \option{oldauthor} and
    \option{oldbookauthor} are ignored.
	\optlist{omiteditor}{false}
	  If set to \wert{true}, the editor is omitted in citations.
	\optlist{origfields}{true}
	  With \option{origfields=true}, the fields \texttt{origlocation} and 
	  \texttt{origdate} (as well as \texttt{origpublisher}, if 
	  \option{nopublisher=false} is given) are printed.
	\optlist{origfieldsformat}{punct}
	  Specifies the appearance of the reprint details (with 
	  \option{origfields=true}): in parentheses, in brackets, or introduced by 
	  \cmd{origfieldspunct} (preset to a comma).
	\optlist[\xbx{footnote-dw} only]{pageref}{false}
	  In addition to the footnote number of the first citation, the page number
	  is referenced.
	\optlist{pagetotal}{false}
	  Whether the field \texttt{pagetotal} is printed or not.
	\optlist{pseudoauthor}{true}
	  If set to \wert{false}, the author of entries with entry option 
		\option{pseudoauthor} are \emph{not} printed.
	\optlist{series}{standard}
	  Position of a work's \texttt{series}: with \wert{standard} as in the 
	  standard styles, with \wert{afteryear} after the \texttt{year}, with \wert{beforeedition} before the \texttt{edition}.
	\optlist{seriesformat}{parens}
    Format of a work's \texttt{series}: with \wert{standard} as in the 
	  standard styles, with \wert{parens} in parentheses.
	\optlist{shorthandibid}{true}
	  Immediately repeated citations of entries with \texttt{shorthand} are 
	  replaced by \enquote{ibid.}
	\optlist{shorthandinbib}{false}
	  If set to \wert{true}, the shorthands will be printed ahead of the bibliography
		entries.
	\optnur[no default]{shorthandwidth}
	  Defines the width of the label in the list of shorthands. Additionally, 
	  after every label the length \cmd{shorthandsep} (the default is 3pt) and 
	  the command \cmd{shorthandpunct} (the default is a colon) are executed.
	\optlist{shortjournal}{false}
    With \option{shortjournal=true} the field \texttt{shortjournal} is used 
    instead of \texttt{journaltitle}. If \texttt{shortjournal} is not set,
    the field \texttt{journaltitle} (and, if available, \texttt{journalsubtitle})
    is used.
	\optlist{singletitle}{false}
    If set to \wert{true}, the title in citations is omitted, unless there is more
		than one work of the same author. This does not apply to full citations.
	\optlist{terselos}{true}
	  A terse version of the list of shorthands is used.
	\optlist{xref}{false}
	  The crossref functionality is used and the option \option{mincrossrefs} is set
		to \wert{1}. See section~\ref{xreffunctionality} on page~\pageref{xreffunctionality} 
		for details.

\subsection{Entry options}
Entry options are set in the field \texttt{options} of an entry in the bib file. 
They may override global options for the respective entry.
	\opt{acronym}
	  The \texttt{shorthands}, with \option{shortjournal=true} also the abbreviated
	  journal titles (\texttt{shortjournal}), are set with the command \cmd{mkbibacro},
	  if the global option \option{acronyms} is set to \wert{true}.
	\opt{citedas}
	  The string \enquote{henceforth cited as} in first citations 
	  (\xbx{authortitle-dw}: with option \option{firstfull} only) of entries 
	  with a \texttt{shorthand} is enforced (\wert{true}) or suppressed 
	  (\wert{false}).
	\opt{oldauthor}
	  The \texttt{author} is not set in the font shape chosen by \option{namefont},
	  if the global option \option{oldauthor} is set to \wert{true}.
	\opt{oldbookauthor}
	  The \texttt{bookauthor} is not set in the font shape chosen by \option{namefont},
	  if the global option \option{oldauthor} is set to \wert{true}.
	\opt{pseudoauthor}
	  The author is printed between \cmd{bibleftpseudo} and \cmd{bibright\-pseudo},
	  if the global option \option{pseudoauthor} is set to \wert{true}. With the
		global option \option{pseudoauthor=false}, the author of entries with the
		entry option \option{pseudoauthor=true} are not printed at all.
	\opt{shorthandibid}
	  Independent of the global option \option{shorthandibid}, the shorthand of 
	  this entry is replaced by \enquote{ibidem} (\wert{true}) or is not 
	  replaced (\wert{false}).

\subsection{\texorpdfstring{\bl}{biblatex} options}
The following list shows \bl{} options which are set to a specific value by \bldw{}. You can find more information on these options in the \bl{} documentation.

	\optset{autocite}{footnote}
	  The command \cmd{autocite} is replaced by \cmd{footcite}.
  \optset{citetracker}{true}
    The \emph{citation tracker} which checks if a work was already cited before
    is activated globally.
	\optset{doi}{false}
	  Whether the field \texttt{doi} is printed or not.
	\optset{eprint}{false}
	  Whether the field \texttt{eprint} is printed or not.
	\optset{ibidtracker}{constrict}
	  In immediately repeated citations of the same work, the citation is
	  replaced by \enquote{ibid.}; text and footnotes are treated separately.
	\optset{idemtracker}{constrict}
	  In immediately repeated citations of the same author, the author's name is
	  replaced by \enquote{idem}; text and footnotes are treated separately.
	\optset{isbn}{false}
	  Whether the fields \texttt{isbn}, \texttt{isrn} and \texttt{issn} are printed or not.
  \optset{loccittracker}{false}
    The \emph{\enquote{loccit} tracker} which checks if the location of a cited
    work is the same as the location last cited (of the same work) is switched
    off.
  \optset{opcittracker}{false}
    The \emph{\enquote{opcit} tracker} which checks if the work is the same as 
    the one which was last cited by the same author is switched off.
  \optset{pagetracker}{true}
    The \emph{page tracker} is switched on; with oneside documents it checks
    for single pages, with twoside documents it checks for double pages 
    (spreads). The internal tests \cmd{iffirstonpage} and \cmd{ifsamepage} 
    use this option.

\subsection{The option \option{journalnumber=date}}
\label{journalnumberdate}
Better than wasting a lot of words in trying to describe the option, I will
rather give some examples which show the \BibTeX\ entry and the corresponding
output with \option{journalnumber=date}. The examples were provided by 
Bernhard Tempel; they are in German, but I think the essentials should be 
clear to the English reader as well.

\begin{lstlisting}
@ARTICLE{Fingiert:1939,
  author = {Anonym},
  title = {Gegen Mißbrauch der Genußgifte},
  journal = {Hannoverscher Kurier},
  volume = {91},
  number = {65},
  issue = {Morgen-Ausg\adddot},
  pages = {2},
  date = {1939-03-06}}
\end{lstlisting}
\fullcite{Fingiert:1939}

\begin{lstlisting}
@ARTICLE{Fingiert:1939a,
  author = {Anonym},
  title = {Gegen Mißbrauch der Genußgifte},
  journal = {Hannoverscher Kurier},
  volume = {91},
  number = {65},
  issue = {Morgen-Ausg\adddot},
  pages = {2},
  date = {1939-03}}
\end{lstlisting}
\fullcite{Fingiert:1939a}

\begin{lstlisting}
@ARTICLE{Gerstmann:2007a,
  author = {Gerstmann, Günter},
  title = {Gerhart Hauptmann-Aktivitäten in Hohenhaus},
  journal = {Schlesischer Kulturspiegel},
  date = {2007},
  volume = {42},
  number = {1},
  pages = {13},
  issue = {Januar--März}}
\end{lstlisting}
\fullcite{Gerstmann:2007a}

\begin{lstlisting}
@ARTICLE{GMG:1939,
  author = {Anonym},
  title = {Gegen Mißbrauch der Genußgifte},
  journal = {Hannoverscher Kurier},
  volume = {91},
  number = {65},
  pages = {2},
  date = {1939-03-06}}
\end{lstlisting}
\fullcite{GMG:1939}

\begin{lstlisting}
@ARTICLE{Guilford:1950,
  author = {Guilford, J[oy] P[aul]},
  title = {Creativity},
  journal = {The American Psychologist},
  date = {1950-09},
  volume = {5},
  number = {9},
  pages = {444--454}}
\end{lstlisting}
\fullcite{Guilford:1950}

\begin{lstlisting}
@ARTICLE{Page:1997,
  author = {Page, Penny Booth},
  title = {E.\,M. Jellinek and the evolution of alcohol studies},
  subtitle = {A critical essay},
  journal = {Addiction},
  date = {1997},
  volume = {92},
  number = {12},
  pages = {1619-1637}}
\end{lstlisting}
\fullcite{Page:1997}

\begin{lstlisting}
@ARTICLE{Fingiert:1939b,
  author = {Anonym},
  title = {Gegen Mißbrauch der Genußgifte},
  journal = {Hannoverscher Kurier},
  number = {65},
  issue = {Morgen-Ausg\adddot},
  pages = {2},
  date = {1939-03-06}}
\end{lstlisting}
\fullcite{Fingiert:1939b}

\begin{lstlisting}
@ARTICLE{Fingiert:1939c,
  author = {Anonym},
  title = {Gegen Mißbrauch der Genußgifte},
  journal = {Hannoverscher Kurier},
  volume = {91},
  issue = {Morgen-Ausg\adddot},
  pages = {2},
  date = {1939-03}}
\end{lstlisting}
\fullcite{Fingiert:1939c}


\begin{lstlisting}
@ARTICLE{Ewers:1906,
  author = {Ewers, Hanns Heinz},
  title = {Rausch und Kunst},
  journal = {Blaubuch},
  date = {1906},
  volume = {1},
  pages = {1726-1730},
  issue = {4. Quartal},
}
\end{lstlisting}
\fullcite{Ewers:1906}

\begin{lstlisting}
@ARTICLE{Fingiert:1939d,
  author = {Anonym},
  title = {Gegen Mißbrauch der Genußgifte},
  journal = {Hannoverscher Kurier},
  volume = {91},
  pages = {2},
  date = {1939-03-19}}
\end{lstlisting}
\fullcite{Fingiert:1939d}

\begin{lstlisting}
@ARTICLE{Fingiert:1939e,
  author = {Anonym},
  title = {Gegen Mißbrauch der Genußgifte},
  journal = {Hannoverscher Kurier},
  volume = {91},
  pages = {2},
  date = {1939-03}}
\end{lstlisting}
\fullcite{Fingiert:1939e}

\begin{lstlisting}
@ARTICLE{Landolt:2000,
  author = {Landolt, H. P. and Borbély, A. A.},
  title = {Alkohol und Schlafstörungen},
  journal = {Therapeutische Umschau},
  date = {2000},
  volume = {57},
  pages = {241-245},
}
\end{lstlisting}
\fullcite{Landolt:2000}

\begin{lstlisting}
@ARTICLE{Chapiro:1930,
  author = {Chapiro, Joseph},
  title = {Das neueste Werk Gerhart Hauptmanns},
  subtitle = {\enquote{Die Spitzhacke}},
  journal = {Neue Freie Presse},
  number = {23773},
  pages = {1-3},
  issue = {Morgenblatt},
  date = {1930-11-19},
}
\end{lstlisting}
\foreignlanguage{german}{\fullcite{Chapiro:1930}}

\begin{lstlisting}
@ARTICLE{Fingiert:1939f,
  author = {Anonym},
  title = {Gegen Mißbrauch der Genußgifte},
  journal = {Hannoverscher Kurier},
  number = {65},
  pages = {2},
  date = {1939-03}}
\end{lstlisting}
\fullcite{Fingiert:1939f}

\begin{lstlisting}
@ARTICLE{Barski:2007,
  author = {Barski, Jacek and Mahnken, Gerhard},
  title = {Museumsverbund Gerhart Hauptmann},
  subtitle = {Ein deutsch-polnisches Kulturprojekt mit Weitblick},
  journal = {Kulturpolitische Mitteilungen},
  date = {2007},
  number = {119},
  pages = {62},
  issue = {IV},
}
\end{lstlisting}
\fullcite{Barski:2007}

\begin{lstlisting}
@ARTICLE{Essig:2005,
  author = {Essig, Rolf-Bernhard},
  title = {Mit liebender Schafsgeduld},
  subtitle = {Erhart Kästner im Dienste Gerhart Hauptmanns},
  journal = {Süddeutsche Zeitung},
  number = {237},
  pages = {16},
  date = {2005-10-14},
}
\end{lstlisting}
\fullcite{Essig:2005}

\begin{lstlisting}
@ARTICLE{Kluwe:2007,
  author = {Kluwe, Sandra},
  title = {Furor poeticus},
  subtitle = {Ansätze zu einer neurophysiologisch fundierten Theorie der literarischen Kreativität am Beispiel der Produktionsästhetik Rilkes und Kafkas},
  journal = {literaturkritik.de},
  date = {2007-02},
  number = {2},
  url = {http://literaturkritik.de/public/rezension.php?rez_id=10438},
}
\end{lstlisting}
\fullcite{Kluwe:2007}

\begin{lstlisting}
@ARTICLE{Burckhardt:2006,
  author = {Burckhardt, Barbara},
  title = {Frauen sind einfach klüger, starke Frauen},
  subtitle = {Michael Thalheimers \enquote{Rose	Bernd} am Hamburger Thalia Theater und Schirin Khodadadians Kasseler Räuber},
  journal = {Theater heute},
  date = {2006},
  number = {5},
  pages = {14-18},
}
\end{lstlisting}
\foreignlanguage{german}{\fullcite{Burckhardt:2006}}

\begin{lstlisting}
@ARTICLE{Ossietzky:1922,
  author = {Ossietzky, Carl von},
  title = {Moritz Heimann \enquote{Armand Carrel} Staatstheater},
  journal = {Berliner Volks-Zeitung},
  date = {1922-03-30},
  issue = {Abend-Ausg\adddot}
}
\end{lstlisting}
\fullcite{Ossietzky:1922}

\begin{lstlisting}
@ARTICLE{Fingiert:1939g,
  author = {Anonym},
  title = {Gegen Mißbrauch der Genußgifte},
  journal = {Hannoverscher Kurier},
  issue = {Abend-Ausgabe},
  pages = {2},
  date = {1939-03}}
\end{lstlisting}
\fullcite{Fingiert:1939g}

\begin{lstlisting}
@ARTICLE{Weiss:1960,
  author = {Weiss, Grigorij},
  title = {Auf der Suche nach der versunkenen Glocke},
  subtitle = {Johannes R. Becher bei Gerhart Hauptmann},
  journal = {Sinn und Form},
  date = {1960},
  pages = {363--385},
  issue = {Zweites Sonderheft Johannes R. Becher},
}
\end{lstlisting}
\fullcite{Weiss:1960}

\begin{lstlisting}
@ARTICLE{Hofer:2006,
  author = {Hofer, Hermann},
  title = {Der Schrei der Verwundeten},
 subtitle = {Erschütternd: Gerhart Hauptmanns \enquote{Rose Bernd} am Hamburger Thalia Theater},
  journal = {Lübecker Nachrichten},
  date = {2006-03-14}}
\end{lstlisting}
\fullcite{Hofer:2006}

\begin{lstlisting}
@ARTICLE{Kammerhoff:2006,
  author = {Kammerhoff, Heiko},
  title = {Rose Bernd},
  journal = {Szene Hamburg},
  date = {2006-04}}
\end{lstlisting}
\fullcite{Kammerhoff:2006}

\begin{lstlisting}
@ARTICLE{Fingiert:1939h,
  author = {Anonym},
  title = {Gegen Mißbrauch der Genußgifte},
  journal = {Hannoverscher Kurier},
  pages = {2},
  date = {1939}}
\end{lstlisting}
\fullcite{Fingiert:1939h}

\section{Commands, bibliography strings, entry types, field formats}
\subsection{Additional commands}
The following list shows additional commands introduced by \bldw{}, along with their standard definitions. These commands can be customized with \cmd{renewcommand}.

%	\befehl{}{}{}
	\befehl{annotationfont}{\cmd{small}\cmd{itshape}}{Font of the field
	  \texttt{annotation}.}
	\befehl{bibfinalnamedelim}{%
	  \cmd{ifnum}\cmd{value}\{liststop\}\textgreater 2\%\\
	  \hspace*{8.1em}\cmd{finalandcomma}\cmd{fi}\%\\
	  \hspace*{8.1em}\cmd{addspace}\cmd{bibstring\{and\}}\cmd{space}}{Final
	    delimiter between names in the bibliography. Compare \bl's
	    \cmd{finalnamedelim}.}
	\befehlleer{bibleftpseudo}{Punctuation after the author when using 
	  \option{pseudoauthor=true}.}
	\befehl{bibmultinamedelim}{\cmd{addcomma}\cmd{space}}{Delimiter between
	  names in the bibliography. Compare \bl's \cmd{multinamedelim}.}
	\befehl{bibrevsdnamedelim}{\cmd{addspace}}{Additional delimiter between
	  first and second name in the bibliography when the scheme 
	  \enquote{Surname, Firstname, Firstname Surname} is given. The comma is not 
	  meant! Compare \bl's \cmd{revsdnamedelim}.}
	\befehlleer{bibrightpseudo}{Punctuation before the author when using 
	  \option{pseudoauthor=true}.}
	\befehl{citefinalnamedelim}{\cmd{slash}}{Final delimiter between names in 
	  citations. Compare \bl's \cmd{finalnamedelim}.}
	\befehl{citemultinamedelim}{\cmd{slash}}{Delimiter between names in 
	  citations. Compare \bl's \cmd{multinamedelim}.}
	\befehlleer{citerevsdnamedelim}{Additional delimiter between
	  first and second name in citations when the scheme 
	  \enquote{Surname, Firstname, Firstname Surname} is given. The comma is not 
	  meant! Compare \bl's \cmd{revsdnamedelim}.}
	\befehl{journumstring}{\cmd{addcomma}\cmd{space}\cmd{bibstring\{number\}}%
	  \cmd{addnbspace}}{Punctuation\slash string ahead of the journal number.}
	\befehl{jourvolnumsep}{\cmd{adddot}}{Punctuation between journal volume and
	  journal number (with \option{journumafteryear=false}).}
	\befehl{jourvolstring}{\cmd{addspace}}{Punctuation\slash string ahead of the 
	  journal volume.}
	\befehl{libraryfont}{\cmd{small}\cmd{sffamily}}{Font of the field
	  \texttt{library}.}
	\befehl{locationdatepunct}{\cmd{addspace}}{Punctuation between \texttt{location}
	  and year (\texttt{year}\slash \texttt{date}) when \option{nopublisher=true} is 
		in use or the publisher is missing.}
	\befehl{locationpublisherpunct}{\cmd{addcolon}\cmd{space}}{Punctuation between \texttt{location}
	  and \texttt{publisher} when \option{nopublisher=false} is in use.}
  \befehl{origfieldspunct}{\cmd{addcomma}\cmd{space}}{Punctuation ahead of 
    the reprint, if the options \option{origfields=true} and
    \option{origfieldsformat=punct} are set.}
	\befehl{publisherdatepunct}{\cmd{addcomma}\cmd{space}}{Punctuation between \texttt{publisher}
	  and year (\texttt{year}\slash \texttt{date}) when \option{nopublisher=false} is in use.}
	\befehlleer{seriespunct}{Punctuation before the \texttt{series}, inside the 
	  parentheses.}
	\befehl{sernumstring}{\cmd{addspace}}{Punctuation\slash string between the 
	  \texttt{series} and its \texttt{number}.}
	\befehl{shorthandinbibpunct}{\cmd{addspace}}{Punctuation after a shorthand in the 
	  bibliography, if \option{shorthandinbib} is used.}
	\befehl{shorthandpunct}{\cmd{addcolon}}{Punctuation after a shorthand in the list
	  of shorthands, if \option{shorthandwidth} is used.}
  \befehl{shorthandsep}{3pt plus 0.5pt minus 0.5pt}{Length between the 
    shorthand and its description, if \option{shorthandwidth} is used.}
  \befehl{textcitesdelim}{\cmd{addspace}\cmd{bibstring\{and\}}\cmd{space}}{Delimiter between
    multiple authors when using \cmd{textcites}.}
  \befehl{titleaddonpunct}{\cmd{addperiod}\cmd{space}}{Punctuation ahead of 
	  \texttt{titleaddon}, \texttt{booktitleaddon} and
	  \texttt{maintitle\-addon}.}
  \befehl{titleyeardelim}{\cmd{addspace}}{Delimiter between
	  \texttt{title} and \texttt{year}, if \option{addyear=true} is set.}

\subsection{Redefined commands}
The following list shows the commands which are defined by \bl{}
and redefined by \bldw{}. These commands can be customized with \cmd{renewcommand}.
  \befehl{labelnamepunct}{\cmd{addcolon}\cmd{space}}{Punctuation after names in the bibliography.}
  \befehl{nametitledelim}{\cmd{addcolon}\cmd{space}}{Delimiter between name and title in citations.}
  \befehl{newunitpunct}{\cmd{addcomma}\cmd{space}}{Punctuation after units in the bibliography.}
  \befehl{subtitlepunct}{\cmd{addperiod}\cmd{space}}{Punctuation between title and subtitle.}
	
\subsection{Additional bibliography strings}
The following list shows the additional bibliography strings introduced by \bldw{}.
There is always a long and a short version. It depends on the \bl{} option \option{abbreviate} which version is used. 
        
\begin{labeling}{mmmmmmm}
  \biblstring{idemdat}{eidem}{eidem}
  \biblstring{idemdatsf}{eidem}{eidem}
  \biblstring{idemdatsm}{eidem}{eidem}
  \biblstring{idemdatsn}{eidem}{eidem}
  \biblstring{idemdatpf}{eisdem}{eisdem}
  \biblstring{idemdatpm}{eisdem}{eisdem}
  \biblstring{idemdatpn}{eisdem}{eisdem}
  \biblstring{idemdatpp}{eisdem}{eisdem}
  \biblstring{inrefstring}{article\cmd{addspace}}{art\cmd{adddotspace}}
\end{labeling}
The bibliography strings can be redefined as follows (but note that you cannot 
define a long and a short version):
\begin{lstlisting}
\DefineBibliographyStrings{english}{%
  thiscite = {at},
  inrefstring = {}}
\end{lstlisting}

\subsection{Redefined bibliography strings}
The following list shows the bibliography strings which are defined by \bl{}
and redefined by \bldw{}. There is always a long and a short version. 
It depends on the \bl{} option \option{abbreviate} which version is used. 
\begin{labeling}{mmmmmmm}
%  \biblstring{}{}{}
  \biblstring{thiscite}{here}{here}
\end{labeling}

\subsection{Entry types}
The following entry types are used in a different way than in \bl.
  \eintragstyp{inreference}{Article in an encyclopedia or dictionary}{incollection}
  \eintragstyp{review}{Review~-- the reviewed work can be referenced with \texttt{xref}}{article}

\subsection{Additional field formats}
The following list shows additional field formats defined by \bldw{}.
        
\feldformat{shorthandinbib}{mkbibbrackets\{\#1\}}{Format of the shorthands
  when using \option{shorthandinbib=true}.}

\section{Further hints}
The following hints are ideas for advanced users to further customize
the styles beyond the options that \bldw{} provides.

\subsection{Delimiter between names}
Other than \bl{}, \bldw\ distinguishes between delimiters that are used in
citations and delimiters that are used in the bibliography. \bl{} has only
\cmd{multinamedelim} (between multiple authors), \cmd{finalnamedelim} (before
the last author) and \cmd{revsdnamedelim} (additional character(s) in
\enquote{Lastname, Firstname\textbar\ and Firstname2 Lastname2}: the \textbar\
indicates the place for the \cmd{revsdnamedelim}).

\bldw{}, however, has \cmd{bibmultinamedelim}, \cmd{bib\-final\-name\-delim}
and \cmd{bibrevsdnamedelim} for the bibliography as well as
\cmd{citemultinamedelim}, \cmd{citefinalnamedelim} and
\cmd{citerevsdna\-medelim} for the citations. Furthermore, \cmd{multinamedelim}, 
\cmd{finalnamedelim} and \cmd{revsdnamedelim} are used in the list of
shorthands. Thus, you can achieve different results. The standard definitions
are as follows:

\begin{lstlisting}[commentstyle=]
\newcommand*{\multinamedelim}{\addcomma\space}
\newcommand*{\finalnamedelim}{%
  \ifnum\value{liststop}>2 \finalandcomma\fi
  \addspace\bibstring{and}\space}
\newcommand*{\revsdnamedelim}{}

\newcommand*{\bibmultinamedelim}{\addcomma\space}
\newcommand*{\bibfinalnamedelim}{%
  \ifnum\value{liststop}>2 \finalandcomma\fi
  \addspace\bibstring{and}\space}%
\newcommand*{\bibrevsdnamedelim}{\addspace}

\newcommand*{\citemultinamedelim}{\slash}
\newcommand*{\citefinalnamedelim}{\slash}
\newcommand*{\citerevsdnamedelim}{}
\end{lstlisting}
When you have multiple authors, they are separated by a slash (/) in citations,
but by comma or (before the last author) by \enquote{and} in the bibliography
and in the list of shorthands. The definitions for the bibliography and for
the list of shorthands are the same as with standard \bl{}. You can customize
these definitions with \cmd{renewcommand*}.

\subsection{Appearance of Shorthands (\cmd{mkbibacro})}
\label{mkbibacro-anpassen}

In \bl{}, acronyms (e.g. \enquote{\textsc{URL}}) are set in small caps.
It uses the command \cmd{mkbibacro} which is defined as:
\begin{lstlisting}
\newcommand*{\mkbibacro}[1]{%
  \ifcsundef{\f@encoding/\f@family/\f@series/sc}
    {#1}
    {\textsc{\MakeLowercase{#1}}}}
\end{lstlisting}
That means: If small caps are available in the used font, acronyms are set
in small caps, otherwise in normal shape.

Typographically, it is better (at least in my view) to use upper case letters
which are slightly letterspaced and scaled down. The letterspacing can be
done with the package \paket{microtype} (if \paket{pdftex} or \paket{pdf\/latex}
are used). The scaling is provided by the package \paket{scalefnt}. Thus, the
command \cmd{mkbibacro} could be redefined in the following way:
\begin{lstlisting}
\usepackage{scalefnt}
\usepackage{microtype}
\renewcommand{\mkbibacro}[1]{%
  \textls[55]{\scalefont{0.95}#1}\isdot}
\end{lstlisting}
The values for \cmd{textls} and \cmd{scalefont} can of course be customized to
your desires or needs.

If a \texttt{shorthand} is an acronym (e.g.\ \enquote{EB} for \emph{Encyclopædia Britannica}),
you can add \texttt{options\,=\,\{acronym=true\}} to the entry and use the global
option \option{acronyms=true}. Then the shorthand will be typeset using the
command \cmd{mkbibacro}. The same applies to abbreviated journals (e.g.
\enquote{PP} for \emph{Past and Present}) using the field \texttt{shortjournal\,=\,\{PP\}}
and \texttt{options\,=\,\{acronym=true\}}.

\end{document}
